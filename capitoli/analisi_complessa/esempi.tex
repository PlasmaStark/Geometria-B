
\chapter{Esempi di funzioni complesse}

\section{Funzione esponenziale}

\begin{definition}
	\label{defn:funzione-esponenziale}
	Definiamo l'esponenziale con la serie di potenze
	\begin{equation}
	\begin{aligned}
		e^z := \sum^{+\infty}_{k=0} \frac{z^k}{k!}
	\end{aligned}
	\end{equation}
\end{definition}

Osserviamo alcune proprietà 

\begin{theorem}
	La funzione esponenziale gode delle seguenti proprietà
	\begin{enumerate}
		\item è olomorfa su $\C$.
		\item $(e^z)' = e^z$.
		\item $e^{i\theta} = \cos \theta + i\sin \theta$ per $\theta \in \R$.
		\item è periodica di periodo $2\pi i$.
		\item $e^z \neq 0$ per ogni $z \in \C$. In particolare non è biunivoca.
	\end{enumerate}
\end{theorem}
\begin{proof}[1]
	Per definizione e per il Teorema \ref{thr:teorema-di-abel}.
\end{proof}
\begin{proof}[2]
	Basta un calcolo esplicito della derivata della serie di potenze
	\begin{equation}
	\begin{aligned}
		(e^z)' & = \sum^{+\infty}_{n=1} n\frac{z^{n-1}}{n!} = \sum^{+\infty}_{n=1} \frac{z^{n-1}}{(n-1)!} \\
				& = \sum^{+\infty}_{p=0} \frac{z^p}{p!} = e^z 
	\end{aligned}
	\end{equation}
\end{proof}
\begin{proof}[3]
	Anche per questo basta un calcolo esplicito
	\begin{equation}
	\begin{aligned}
		e^{iz} & = \sum^{+\infty}_{n=0} \frac{(iz)^{n}}{n!} = \sum^{+\infty}_{k=0} \frac{(iz)^{2k}}{2k!} + \sum^{+\infty}_{k=0} \frac{(iz)^{2k+1}}{2k+1!}\\
			& = \sum^{+\infty}_{k=0} (-1)^k\frac{z^{2k}}{2k!} + i\sum^{+\infty}_{k=0} (-1)^k\frac{z^{2k+1}}{2k+1!}\\
			& = \cos z + i \sin z
	\end{aligned}
	\end{equation}
\end{proof}
\begin{proof}[4]
	Basta osservare che $e^{a + 2\pi i} = e^a e^{2\pi i } = e^a$, il fatto che $e^{2\pi i} = 1$ è naturale dal punto precedente.
\end{proof}
\begin{proof}[5]
	Sia $\alpha \in \C$, allora $e^z = \alpha$ vuol dire che se $z = x+iy$ allora 
	\begin{equation}
	\begin{cases}
		x = \ln |\alpha|\\
		y = \arg \alpha
	\end{cases}
	\end{equation}
	per cui se $\alpha = 0$, $x$ non sarebbe ben definito. Mentre per tutti gli altri valore di $\alpha \in \C$ è definita.
\end{proof}

\section{Funzioni trigonometriche}
	
	\begin{definition}
		\label{defn:sin-cos}
		Definiamo le funzioni trigonometriche in funzione delle serie di potenze come segue
		\begin{equation}
		\begin{aligned}
			\sin(z) := \sum^{+\infty}_{k=0} (-1)^k\frac{z^{2k+1}}{(2k+1)!} \;\ \cos(z) := \sum^{+\infty}_{k=0} (-1)^k\frac{z^{2k}}{(2k)!}
		\end{aligned}
		\end{equation}
	\end{definition}
	
	\begin{theorem}
		Valgono i seguenti fatti sulle funzioni trigonometriche
		\begin{enumerate}
			\item 
			\begin{equation}
			\begin{aligned}
				\sin(z) = \frac{e^{iz} - e^{-iz}}{2} \;\ \cos(z) = \frac{e^{iz} + e^{-iz}}{2}
			\end{aligned}
			\end{equation}
			\item sono $2\pi$ periodiche.
			\item non sono limitate.
			\item l'immagine di $\sin$ e $\cos$ coincide con $\C$.
			\item le seguenti funzioni sono olomorfe tranne nei loro rispettivi poli
			\begin{equation}
			\begin{aligned}
				\tan z & := \frac{\sin z}{\cos z} \in \mathcal{O}(\C \setminus \{ \pi /2 + k\pi \mid k \in \N \}) \\
				\cot z & := \frac{\cos z}{\sin z} \in \mathcal{O}(\C \setminus \{k\pi \mid k \in \N \})
			\end{aligned}
			\end{equation}
		\end{enumerate}
	\end{theorem}
	\begin{proof}[1]
		% TODO
	\end{proof}
	\begin{proof}[2]
		% TODO: DIMOSTRAZIONE
	\end{proof}
	\begin{proof}[3]
		% TODO: DIMOSTRAZIONE
	\end{proof}
	\begin{proof}[4]
		% TODO: DIMOSTRAZIONE
	\end{proof}
	\begin{proof}[5]
		% TODO: DIMOSTRAZIONE
	\end{proof}
	
\section{Funzione logaritmo}
	
	\begin{definition}
		\label{defn:logaritmo-principale}
		Il \textbf{logaritmo principale} è la funzione 
		\begin{equation}
		\begin{aligned}	
			\morphism{\log}{\C \setminus \{x \in \R \mid x \le 0\}}{\C}
		\end{aligned}
		\end{equation}
		definita da $\log(z) = \ln(|z|) + i\arg z$ con $\arg z \in \left(-\pi,\pi\right]$ 
	\end{definition}

	\begin{remark}
		In generale si perdono molte delle proprietà del logaritmo sui reali. Per esempio $\log(xy) \neq \log(x)+\log(y)$.
	\end{remark}

	\begin{theorem}
		La funzione logaritmo princiaple è olomorfa e la sua derivata è $\log'(z) = 1/z$.
	\end{theorem}
	\begin{proof}
		Dimostriamo che è olomorfa per le equazioni di Cauchy-Riemann, per cui ponendo
		\begin{equation}
		\begin{aligned}
			\log(x+iy) = \frac{1}{2}\ln(x^2 + y^2) + i\arctan(y/x) = u(x,y) + iv(x,y)
		\end{aligned}
		\end{equation}
		da cui le derivate ci danno le relazioni cercate.\\
		
		Sempre per le equazioni di Cauchy-Riemann otteniamo che $\log(z) = u_x + iv_x = 1/z$.
	\end{proof}
	\begin{remark}
		Osserviamo infine che poiché la funzione argomento può prendere valori con periodicità di $2\pi$, abbiamo effettivamente infinite definizioni della funzione logaritmo. Questo può presentare alcuni vantaggi. Infatti la funzione radice quadrata 
		$f_0(z) = e^{\log(z)/2} = \sqrt{|z|}e^{i(\arg z)/2}$, ma se si usa un'altra definizione di logaritmo, ovvero consideriamo un'altra possibile branca o determinazioni possibili della radice quadrata. Allora otteniamo $f_1(z) = e^{\log'(z)/2} = \sqrt{|z|}e^{i\arg(z)/2 + i\pi}$ che rappresenta la radice `negativa'. E così via per molte altre funzioni.
	\end{remark}